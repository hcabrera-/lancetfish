% the abstract

El desarrollo de nuevos procesos para la manufactura de circuitos integrados ha abierto la posibilidad de la fabricacion de sistemas digitales contenidos en un solo encapsulado, comúnmente nombrados sistemas en-chip o SoC por sus siglas en ingles (\textit{System on-Chip}). El beneficio de la fabricación de circuitos con alta densidad de transistores ha impactado el diseño de sistemas para dispositivos reconfigurables, abriendo la posibilidad de explorar metodologías de diseño que anteriormente resultaban prohibitivas por los altos requerimientos de elementos lógicos.

La aplicación de los fundamentos del diseño SoC en dispositivos reconfigurables busca extrapolar arquitecturas o conceptos probados en dispositivos de aplicación específica para generar mayores rendimientos en sistemas implementados en FPGAs. Sin embargo, la diferencia en la naturaleza de la tecnología objetivo puede diluir los beneficios si no se toma en consideración las restricciones propias de los dispositivos reconfigurables. El diseño de sistemas en-chip compuestos por múltiples elementos de procesamiento es un área que refleja de manera más drástica la disminución de ganancias de rendimiento al implementar de forma directa la experiencia de su contraparte de propósito específico.

En este documento se presenta una nueva estrategia para la asignación de cargas de trabajo en aceleradores basados en múltiples núcleos de procesamiento, así como una infraestructura de comunicación para el diseño de aceleradores en hardware reconfigurable basados en esta estrategia. La arquitectura desarrollada está compuesta de una \textit{red en-chip}, también referida como \textit{NoC - Network on-Chip}, diseñada tomando en cuenta las restricciones que impone los elementos reconfigurables de un dispositivo FPGA, así como las cargas de trabajo propias de un acelerador en hardware de propósito especifico. La red provee de mecanismos para la inclusión de nuevos elementos de procesamiento, ampliando el espacio de aplicaciones al cual puede dar servicio.