% the abstract

El desarrollo de nuevos procesos para la manufactura de circuitos integrados ha abierto la posibilidad del desarrollo de sistemas digitales contenidos en un solo encapsulado, comúnmente nombrados \textit{SoC - System on-Chip} o sistemas en-chip. El beneficio de la fabricación de circuitos con alta densidad de transistores ha impactado el diseño de sistemas para dispositivos reconfigurables, abriendo la posibilidad de explorar metodologías de diseño que anteriormente resultaban prohibitivas por los altos requerimientos de elementos lógicos.

La aplicación de los fundamentos del diseño SoC en dispositivos reconfigurables busca extrapolar arquitecturas o conceptos probados en dispositivos de aplicación específica para generar mayores rendimientos en sistemas implementados en FPGAs. Sin embargo, la diferencia en la naturaleza de la tecnología objetivo puede diluir los beneficios si no se toma en consideración las restricciones propias de los dispositivos reconfigurables. El diseño de sistemas en-chip compuestos por múltiples elementos de procesamiento es un área que refleja de manera más drástica la disminución de ganancias de rendimiento al implementar de forma directa la experiencia de su contraparte de propósito específico.

En este documento se presenta una propuesta de infraestructura para el diseño de aceleradores en hardware reconfigurable basados en múltiples núcleos de procesamiento. La arquitectura desarrollada está compuesta de una \textit{red en-chip}, también referida como \textit{NoC - Network on-Chip}, diseñada tomando en cuenta las restricciones que impone los elementos reconfigurables de un dispositivo FPGA, así como las cargas de trabajo propias de un acelerador en hardware de propósito especifico. La red provee de mecanismo para la inclusión de nuevos elementos de procesamiento, ampliando el espacio de aplicaciones al cual puede dar servicio.

\clearpage

The development of new integrated circuit fabrication processes has opened the possibility for the design of full digital systems deployed on a single chip. This kind of designs are commonly referred as System on-Chip (SoC). The fabrication of high transistor density integrated circuits has impacted positively the manufacturing process for reconfigurable devices, opening the possibility for explore new design methodologies which were considered impossible to deploy on this kind of circuits due the high demand on logic elements.

The use of SoC design fundamentals when targeting reconfigurable devices, tries to extrapolate proven architectures previously implemented on application specific integrated circuits on reconfigurable logic. However, the difference on how the logic is implemented physically on reconfigurable devices diluted the benefits shown on other target technologies. The nature of the reconfigurable fabric must be taken on account when using architectures developed for ASIC devices. Systems on-chip containing multiple processing cores are prone to show drastic demises of performance when architectures designed for ASIC devices are used for implementations on reconfigurable devices.

This document presents a communication infrastructure proposal for the design of hardware accelerators, targeting reconfigurable devices, based on multiple processing cores. The developed infrastructure is based on a Network On-Chip(NoC) that has been designed taking in account the constrains impossed by the reconfigurable fabric of the FPGA devices, and the nature of tipical workloads for single porpouse harwdware accelerators. The network provides a simple interface for the inclusion of new processing elements, broadening the application space of this work.