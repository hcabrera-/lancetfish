The development of new integrated circuit fabrication processes has opened the possibility for the design of full digital systems deployed on a single chip. This kind of designs are commonly referred as System on-Chip (SoC). The fabrication of high transistor density integrated circuits has impacted positively the manufacturing process for reconfigurable devices, opening the exploration of new design methodologies which were considered impossible to set up on this kind of circuits due the high demand on logic elements.

Extrapolating architectures previously implemented on application specific integrated circuits to reconfigurable logic is one commun mistake while using SoC design fundamentals. The difference on how the logic is deployed physically on a reconfigurable device diluted the benefits shown on other target technologies. The nature of the reconfigurable fabric must be taken on account when using architectures developed for ASIC devices. Systems on-chip containing multiple processing cores are prone to show drastic demises of performance when architectures designed for ASIC devices are used for implementations on reconfigurable devices.

This document presents a workload distribution strategy proposal for hardware accelerators based on multiple processing cores. Also, a communication infrastructure architecture for the design of reconfigurable hardware accelerators based on the proposed strategy is presented. The developed infrastructure is based on a Network On-Chip(NoC) that has been designed taking in account the constrains impossed by the reconfigurable fabric of the FPGA devices, and the nature of tipical \mbox{workloads} for single porpouse harwdware accelerators. The network provides a simple interface for the inclusion of new processing elements, broadening the application space of the work presented on this document.